% ----------------------------------------------------------
% Fundamentação Teórica
% ----------------------------------------------------------
\chapter[Fundamentação Teórica]{Fundamentação Teórica}

Neste capítulo é apresentado os conceitos básicos necessários para compreensão deste trabalho bem como os trabalhos relacionados.

% --------------------------%
% --- Conceitos Básicos --- %
\section{Conceitos Básicos}

Esta seção apresenta os conceitos teóricos relacionados ...

\subsection{Exemplos}

Exemplo de inserção de figura (Figura~\ref{fig:figura}).
\begin{figure}[htb]
    \centering
    \includegraphics[width=10cm]{figuras/Heckert_GNU_white}
    \caption{Exemplo de figura.}
    \label{fig:figura}
\end{figure}

Exemplo tabela (Figura~\ref{tab:tabela}).
\begin{table}[htb]
    \centering
    \caption{Exemplo de tabela.}
    \begin{tabular}{c|cc}
        \hline
        Algoritmo & Resultado & Tempo\\
        \hline
        Alg 1 & $10,5$ & $0,2$\\
        Alg 2 & ... & ...\\
        ... & ... & ...\\
        \hline
    \end{tabular}
    \label{tab:tabela}
\end{table}

Exemplo de equação (Equação~\eqref{eq:equacao}).
\begin{equation}
x=\frac{-b\pm\sqrt{b^2-4ac}}{2a}
\label{eq:equacao}
\end{equation}

Além disso, quando formos escrever números ou símbolos matemáticos ao longo do texto, usamos a seguinte notação: $x=2$, $y^2=\log_2(x)$, etc.

Exemplo de algoritmo, usando o pacote \texttt{algorithm2e} (Algoritmo~\ref{alg:algoritmo})
\begin{algorithm}
    \caption{Exemplo de algoritmo.}
    \Entrada{$x$ e $y$ inteiros.}
    \Saida{$z$ inteiro.}
    \Se{$x>y$}{
    	z = x - y\;
    }
    \label{alg:algoritmo}
\end{algorithm}


% -------------------------------%
% --- Trabalhos Relacionados --- %
\section{Trabalhos Relacionados}


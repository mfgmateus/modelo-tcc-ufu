% ------------------------------------------------------------------------
% PROPRIEDADES DO DOCUMENTO
% ------------------------------------------------------------------------
\documentclass[12pt,
openright, 
oneside, %
%twoside, %TCC: Se seu texto tem mais de 100 páginas, descomente esta linha e comente a anterior
a4paper,    %
%english,   %
brazil]{facom-ufu-abntex2}

% ------------------------------------------------------------------------
% PACOTES
% ------------------------------------------------------------------------
\usepackage{amsmath}
\usepackage{icomma}
\usepackage[portuguese,ruled]{algorithm2e}

% ------------------------------------------------------------------------
% INFO para CAPA e FOLHA DE ROSTO 
% ------------------------------------------------------------------------
\titulo{Título do Trabalho de Conclusão de Curso} %TCC

\autor{Nome completo do autor} %TCC
\data{Ano de defesa}

\orientador{Nome completo do orientador} %TCC
%\coorientador{Nome completo do orientador caso tenha} %TCC

\begin{document}
\frenchspacing 

% ----------------------------------------------------------
% ELEMENTOS PRÉ-TEXTUAIS
% ----------------------------------------------------------
%\pretextual
\imprimircapa
\imprimirfolhaderosto


% --- Inserir folha de aprovação --- %
\begin{folhadeaprovacao}

  \begin{center}
    {\ABNTEXchapterfont\large\imprimirautor}

    \vspace*{\fill}\vspace*{\fill}
    {\ABNTEXchapterfont\bfseries\Large\imprimirtitulo}
    \vspace*{\fill}
    
    \hspace{.45\textwidth}
    \begin{minipage}{.5\textwidth}
        \imprimirpreambulo
    \end{minipage}%
    \vspace*{\fill}
   \end{center}
    
   Trabalho aprovado. \imprimirlocal, 24 de novembro de 2012: %TCC:

   \assinatura{\textbf{\imprimirorientador} \\ Orientador}  
   \assinatura{\textbf{Professor}}% \\ Convidado 1} %TCC:
   \assinatura{\textbf{Professor}}% \\ Convidado 2} %TCC:
   %\assinatura{\textbf{Professor} \\ Convidado 3}
   %\assinatura{\textbf{Professor} \\ Convidado 4}
      
   \begin{center}
    \vspace*{0.5cm}
    {\large\imprimirlocal}
    \par
    {\large\imprimirdata}
    \vspace*{1cm}
  \end{center}
  
\end{folhadeaprovacao}
% \includepdf{folhadeaprovacao_final.pdf} % TCC: depois de aprovado o trabalho, descomente esta linha e comente a anterior para incluir o scan da folha de aprovação.

%% OBS.: as seções dedicatória, agradecimento e epígrafe não são obrigatórias. Só as mantenha se achar pertinente.

% --- Dedicatória --- %
\imprimirdedicatoria{Insere o texto de dedicatória ... }

% --- Agradecimentos --- %
\imprimiragradecimentos{Insere o texto de agradecimentos ...}

% --- Epígrafe --- %
\imprimirepigrafe{Insere alguma citação que ache conveniente caso queira ...}
	
% --- Resumo em português --- %
\begin{resumo} %TCC:
Segundo a \citeonline[3.1-3.2]{NBR6028:2003}, o resumo deve ressaltar o objetivo, o método, os resultados e as conclusões do documento. A ordem e a extensão  destes itens dependem do tipo de resumo (informativo ou indicativo) e do tratamento que cada item recebe no documento original. O resumo deve ser precedido da referência do documento, com exceção do resumo inserido no  próprio documento. (\ldots) As palavras-chave devem figurar logo abaixo do resumo, antecedidas da expressão Palavras-chave:, separadas entre si por ponto e finalizadas também por ponto.
 
 \vspace{\onelineskip}
 \noindent
 \textbf{Palavras-chave}: Até, cinco, palavras-chave, separadas, por, vírgulas. %TCC:
\end{resumo}

% --- lista de ilustrações --- %
\listailustracoes

% --- lista de tabelas --- %
\listatabelas

% --- lista de abreviaturas e siglas --- %
\input{preTextuais/abreviaturas}

% --- lista de símbolos --- %
\input{preTextuais/simbolos} % caso não existe símbolos no trabalho, comente esta linha

% --- sumario --- %
\sumario


% ----------------------------------------------------------
% ELEMENTOS TEXTUAIS
% ----------------------------------------------------------
\textual

% --- Introdução --- %
% ----------------------------------------------------------
% Introdução
% ----------------------------------------------------------
\chapter[Introdução]{Introdução}

Contextualização, problema, hipótese, ... \cite{EIA649B}.

% ------------------%
% --- Objetivos --- %
\section{Objetivos}
Descrever o objetivo geral e os objetivos específicos.

% ---------------%
% --- Método --- %
\section{Método}
Descrever o método utilizado.

% -------------------%
% --- Resultados --- %
\section{Resultados}
Descrever os resultados de seu trabalho.

% --------------------------------%
% --- Organização do Trabalho --- %
\section{Organização do Trabalho}
Descrever a organização do trabalho.

% ---  Fundamentação teórica --- %
% ----------------------------------------------------------
% Fundamentação Teórica
% ----------------------------------------------------------
\chapter[Fundamentação Teórica]{Fundamentação Teórica}

Neste capítulo é apresentado os conceitos básicos necessários para compreensão deste trabalho bem como os trabalhos relacionados.

% --------------------------%
% --- Conceitos Básicos --- %
\section{Conceitos Básicos}

Esta seção apresenta os conceitos teóricos relacionados ...

\subsection{Exemplos}

Exemplo de inserção de figura (Figura~\ref{fig:figura}).
\begin{figure}[htb]
    \centering
    \includegraphics[width=10cm]{figuras/Heckert_GNU_white}
    \caption{Exemplo de figura.}
    \label{fig:figura}
\end{figure}

Exemplo tabela (Figura~\ref{tab:tabela}).
\begin{table}[htb]
    \centering
    \caption{Exemplo de tabela.}
    \begin{tabular}{c|cc}
        \hline
        Algoritmo & Resultado & Tempo\\
        \hline
        Alg 1 & $10,5$ & $0,2$\\
        Alg 2 & ... & ...\\
        ... & ... & ...\\
        \hline
    \end{tabular}
    \label{tab:tabela}
\end{table}

Exemplo de equação (Equação~\eqref{eq:equacao}).
\begin{equation}
x=\frac{-b\pm\sqrt{b^2-4ac}}{2a}
\label{eq:equacao}
\end{equation}

Além disso, quando formos escrever números ou símbolos matemáticos ao longo do texto, usamos a seguinte notação: $x=2$, $y^2=\log_2(x)$, etc.

Exemplo de algoritmo, usando o pacote \texttt{algorithm2e} (Algoritmo~\ref{alg:algoritmo})
\begin{algorithm}
    \caption{Exemplo de algoritmo.}
    \Entrada{$x$ e $y$ inteiros.}
    \Saida{$z$ inteiro.}
    \Se{$x>y$}{
    	z = x - y\;
    }
    \label{alg:algoritmo}
\end{algorithm}


% -------------------------------%
% --- Trabalhos Relacionados --- %
\section{Trabalhos Relacionados}



% --- Desenvolvimento --- %
% ----------------------------------------------------------
% Desenvolvimento
% ----------------------------------------------------------
\chapter[Desenvolvimento]{Desenvolvimento}

Um ou mais capítulos (por exemplo um para experimentos).

% --- Conclusão --- %
% ----------------------------------------------------------
% Conclusão
% ----------------------------------------------------------
\chapter[Conclusão]{Conclusão}

Descrever aqui as conclusões e/ou considerações finais.

Destacar as contribuições originais do trabalho.

Propor trabalhos futuros em continuidade ao trabalho realizado.

% ----------------------------------------------------------
% ELEMENTOS PÓS-TEXTUAIS
% ----------------------------------------------------------
\postextual

% --- Referências bibliográficas --- %
\bibliography{blibiografia}

% --- Apêndices --- %
% só mantenha se for pertinente.
\begin{apendicesenv}
\partapendices % Imprime uma página indicando o início dos apêndices

% --- Apendice 1--- %
\include{apendices/apendice1}

\end{apendicesenv}

% --- Anexos --- %
% so mantenha se pertinente.
\begin{anexosenv}
\partanexos % Imprime uma página indicando o início dos anexos

% --- Anexo 1 --- %
\include{anexos/anexo1}

\end{anexosenv}


% ------------------------------------------------------------------------
% FIM DO DOCUMENTO
% ------------------------------------------------------------------------
\printindex
\end{document}